\documentclass[12pt, a4paper]{report}

\usepackage[english]{babel}
\usepackage{amsmath}
\usepackage{graphicx}
\usepackage{xcolor}
\usepackage[hidelinks]{hyperref}
\usepackage{xltabular}
\usepackage{float}
\usepackage{titlesec}
\usepackage{hhline}
\usepackage{menukeys}
\usepackage{dsfont}

\hypersetup{
    colorlinks=true,
    urlcolor=blue,
		linkcolor=black
}

\titleformat{\chapter}[block]
{\normalfont\Huge\bfseries}{\thechapter\ }{0pt}{\Huge}
\titlespacing*{\chapter}{0pt}{-2cm}{10pt}

\usepackage[indent=0pt]{parskip}

\begin{document}
\pagenumbering{gobble}

\begin{center}
    \Large{\textbf{University of West Bohemia in Pilsen\\Faculty of Applied Science}}

    \vspace{\fill}
    \normalsize{Semestral work:\\}
    \Large{\textbf{SeedSeeker}}

    \vspace{\fill}
    \normalsize{Authors: Pavel Altmann, Jakub Kupčik,\\Matěj Bartička, David Wimmer, Patrik Holub\\Date: 1st July 2025}
\end{center}
\newpage

\pagenumbering{arabic}
\setcounter{page}{2}
\tableofcontents
\newpage

\chapter{Description}
In programming, generating random values is a common requirement, particularly for applications such as 
cryptography and systems requiring unpredictability (e.g., slot machines). If an attacker could predict 
these values, they could exploit this knowledge for malicious purposes.

Most systems employ Pseudo-Random Number Generators (PRNGs), which produce sequences of values through 
mathematical algorithms rather than true randomness. These algorithms generate sequences that satisfy specific 
statistical distributions (typically uniform distribution) and share a crucial characteristic: each value depends 
on previous values. Consequently, the entire sequence derives from an initial starting point called the seed.

Knowledge of the seed enables complete sequence prediction. While methods exist to deduce seeds for certain 
algorithms, no comprehensive tool previously implemented these techniques. Our solution addresses this gap.

We developed SeedSeeker, a Python application that attempts to reverse-engineer both the generator algorithm and 
its seed from output sequences. The implementation adopts a modular library approach, facilitating the 
addition of new generators and reversal methods as they become available.

\chapter{User Manual}
\section{Download}
SeedSeeker is available at \href{https://github.com/wren-projects/SeedSeeker}{GitHub} 
(\url{https://github.com/wren-projects/SeedSeeker}). Users have two installation options:

\begin{itemize}
    \item Download pre-built binaries from the Releases section (recommended for users without Python experience)
    \item Clone the repository and run directly via Python (requires dependency installation)
\end{itemize}

For Python execution, required dependencies can be installed manually or via tools like 
\href{https://github.com/astral-sh/uv}{uv} or \href{https://python-poetry.org}{Poetry}.

\section{Running}
The tool's help page (accessible via command-line flags) and manpage provide comprehensive usage instructions. 
SeedSeeker operates in three primary modes:

\begin{enumerate}
    \item \textbf{Generation}: Execute a specified generator with a given seed 
        (see Table~\ref{table:1} for seed formats)
    \item \textbf{Reversal}: Attempt to deduce the initial state 
        from output sequences using all applicable reversal algorithms
    \item \textbf{Prediction}: Generate future values from a known initial state
\end{enumerate}

\textbf{Important Limitations:}
\begin{itemize}
    \item Reversal algorithms perform optimally with standard generator implementations
    \item Degenerate states (e.g., constant output sequences) may produce incorrect predictions
    \item Successful reversal doesn't guarantee accurate future predictions for all generators
\end{itemize}

\begin{table}
\begin{xltabular}{\textwidth}{| >{\raggedright\arraybackslash}X | >{\raggedright\arraybackslash}X | >{\raggedright\arraybackslash}X |}
	\caption{Seed formats for generators}
	\label{table:1}\\
	\hline
	\bf{Generator} & \bf{Seed Format} & \bf{Conditions} \\
	\hhline{|=|=|=|}
	LCG & 
	"$m;a;c;x_0$" & 
	$m > 0 \wedge\break 0 < a < m \wedge\break 0\leq c \leq m \wedge\break 0 \leq x_0 < m$\\
	\hline
	Lagged Fibonacci & 
	"$r;s;m;seed$" or \break "$r;s;m;s_0, s_1, ...$" & 
	$m > 0 \wedge\break 0 < r < m \wedge\break 0 < s < m \wedge\break r \neq s \wedge\break\break seed > 0 \vee\break
	s_0, s_1, ... \geq 0 \wedge\break len(s_i) = max(r, s)$\\
	\hline
	ran3 &
	"$seed$" & 
  $seed \in \mathds{R}$\\
	\hline
	Xoshiro256** &
	"$a;b;c;d$" & 
	$a,b,c,d \ge 0 \wedge\break 0 < a,b,c,d \leq 2^{64}$ \\
	\hline
	Mersenne Twister&
	"$seed$" &
	$0 \leq seed \leq 2^{32}$ \\
	\hline
\end{xltabular}
\end{table}

\chapter{Reversal Algorithms}
\section{LCG}
Let $x_i$ be the sequence of LCG output. The sequence is given by the recursive formula:
$$x_{i+1} = (a \cdot x_i + c) \mod M$$

We first create a second sequence:
$$y_{i} = x_{i+1} - x_{i}$$ Note $y_{i+1} = a \cdot y_{i} \mod M$.

Let us denote $$r_i = y_{i} \cdot y_{i+3} - y_{i+1} \cdot y_{i+2}$$
And note $r_i \equiv 0 \mod M$.

So to retrieve the value of modulo, we can just take an arbitrary number of $r_i$s, calculate their GCD and estimate $M$ based on that value.

Once we have modulo, we can just multiply $r_{i+1}$ by the inverse of $r_{i}$ in modulus $M$ to retrieve $a$. Finally we trivially retrieve $c$.

\section{ran3}
The output sequence directly reveals the initial state. By observing 55 consecutive values, we completely determine the generator's state.

\section{Lagged Fibonacci}
Let $x_i$ be the sequence of outputs. We know, that $x_{i} = ((x_{i-r} + x_{i-s}) \mod M) + K$, where $K$ is one if $x_{i-1} > x_{i-r} + x_{i-s}$.
We reverse this generator by brute force.

We try out all possible values of $r$ and $s$ (lower than some threshold). Then, for each of them, we calculate $x_{i} - x_{i-r} - x_{i-s}$. We know this is either $M-1, M, M+1$ or $0$. By doing this multiple times, we pinpoint get the exact value of $M$.

\section{Xoshiro256**}
This generator uses a total of 256 bits of inner state broken up into 4 \verb|uint_64| variables (stored in array `s`).

For full description, see \href{https://en.m.wikipedia.org/wiki/Xorshift#xoshiro256**}{here}.

Let us explore, how does the inner state change in one iteration:

$$[A, B, C, D] \to [A\oplus B\oplus D, A\oplus B \oplus C, A\oplus C\oplus (B<<17), \text{rotl}(B\oplus D)]$$
, where $\oplus$ denotes bitwise XOR.

Note that the output is always reversible into the second element of given state.

We can retrieve \verb|s[1]| and \verb|s[0]^s[1]^s[2]| from the first 2 outputs.

Let us denote t the state after one iteration. Note we know the value of \verb|t[1]| and \verb|t[2]|.

From 3rd output, we can reconstruct \verb|t[0]|.
\section{Mersenne Twister}
We incorporated the \href{https://github.com/tna0y/Python-random-module-cracker}{randcrack} library for Mersenne Twister reversal. Refer to the source repository for implementation details.

\chapter{Library Description}
SeedSeeker's architecture facilitates extensibility through a modular library system. Adding new functionality requires:

\begin{enumerate}
    \item Implementing a \texttt{GeneratorState} class
    \item Implementing a \texttt{Generator} class
    \item Developing a reversal method
    \item Updating the import registry
\end{enumerate}

The LCG implementation serves as a reference example. Import configurations reside in the CLI directory.
\end{document}
